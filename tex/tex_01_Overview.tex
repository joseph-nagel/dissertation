\begin{refsection}

\section{Motivation}
% UNCERTAINTY QUANTIFICATION
The increasing sophistication of computer simulations for predicting the behavior of physical systems necessitates the specification of a growing number of model parameters.
This motivates the engagement in both forward and inverse uncertainty quantification.
% FORWARD PROPAGATION
One can represent the degree as to which the model input parameters are not precisely known as a probability distribution.
Either this may reflect a lack of knowledge about the true parameter value or a natural variability of the input realizations.
The stochasticity in the model parameters then induces randomness in the model predictions, the quantification of which is the chief goal of uncertainty forward propagation.
\par % BAYESIAN INFERENCE
While uncertainty propagation deals with the characterization of the model response for a given input distribution,
inverse uncertainty quantification aims at the indirect determination of the actual distribution of the uncertain inputs with experimental measurements of the outputs.
In Bayesian inverse problems the epistemic uncertainty of the constant but unknown forward model parameters is translated into the prior distribution and probabilistically updated.
The resulting posterior distribution encodes the reduced level of epistemic uncertainty that remains after integrating the information yielded by the data.
Point estimates of the parameters and predictive distributions of future outcomes can then be derived.
\par % KEY CHALLENGE: ALEATORY VARIABILITY
The Bayesian approach to inverse problems does not only allow for improving one's knowledge about the fixed yet unknown parameters and growing one's confidence in the predictions,
it also measures the uncertainty in the model input estimation and output prediction.
Therefore it gains advantage over deterministic solutions to inverse problems.
A limitation of the approach is that it lacks the possibility to manage the aleatory uncertainty of genuinely random quantities that vary during the experimentation.
Nuisance variables that merely complicate the analysis or aleatory variables whose distribution is of inferential interest are examples of such quantities.
They are incorrectly treated as constants in current practice.
\par % KEY CHALLENGE: COMPUTATIONAL COST
In addition to the unanswered question of how aleatory variability might be handled,
another limiting factor of Bayesian inversion is the expense of computing the posterior distribution numerically.
One of the very few serviceable tools for that purpose is Markov chain Monte Carlo sampling.
This technique suffers from the absence of a clear convergence criterion and the autocorrelation of the obtained posterior samples.
It demands an excessive number of serial forward model runs which may easily exceed the available computational budget.
This prompts researchers and practitioners to implement advanced sampling algorithms and to find completely new solutions.

\section{Contribution}
% KEY CHALLENGES
The fact that aleatory variability is ignored, the need for more efficient sampling schemes and the lack of fundamental alternatives
form obstacles to Bayesian inverse problems in complex applications.
% CORE CONTRIBUTIONS
In this dissertation it is tried to overcome these difficulties.
The core contributions are concisely summarized as follows.
\begin{enumerate}[label=\arabic*)]
  \item \label{item:CoreContribution:1} A unifying framework for the management of aleatory uncertainty in inverse problems is developed.
  \item \label{item:CoreContribution:2} Hamiltonian Monte Carlo is promoted for efficient posterior exploration in high-dimensional spaces.
  \item \label{item:CoreContribution:3} Novel approaches for computational Bayesian inference and posterior characterization are proposed.
  \item \label{item:CoreContribution:4} Complex inverse problems from structural, mechanical and also hydrological engineering are solved.
\end{enumerate}
\par % SHORT DESCRIPTION
First, the developed framework in \ref{item:CoreContribution:1} allows one to master inverse problems in the presence of epistemic uncertainty and aleatory variability.
Unknown parameters can be identified along with the distribution of aleatory variables.
Second, in \ref{item:CoreContribution:2} the computational cost of sampling high-dimensional posteriors is drastically reduced by Hamiltonian Monte Carlo.
This is a general-purpose Markov chain Monte Carlo method which proves especially beneficial to the previously devised framework for inversion under polymorphic uncertainty.
Third, spectral Bayesian inference is proposed as a radically different technique for computing the posterior density in \ref{item:CoreContribution:3}.
It rests on spectral likelihood expansions and enables semi-analytic and sampling-free Bayesian inference.
Another recently emerged method based on optimal transportation theory is also investigated and compared to spectral inference.
\par % APPLICATIONS
A wide range of practical engineering problems can be addressed with the new methodological developments.
On the one hand, simple problems with simulated data serve for prototyping and benchmarking purposes.
Bayesian inversion under multiple types of uncertainty and Hamiltonian Monte Carlo are both applied to the estimation of the material variability throughout an ensemble of structural elements.
The inverse heat conduction problem posed by calibrating the thermal properties of a composite material with temperature measurements is used to demonstrate the newly devised schemes of inference.
On the other hand, in \ref{item:CoreContribution:4} some more interesting problems involving real data and realistic models are solved.
This includes the NASA Langley multidisciplinary uncertainty quantification challenge, the probabilistic assessment of structural masonry,
and the Bayesian calibration of a hydrological urban drainage simulator.
\par % PUBLICATIONS
The methodological progress achieved and the real problems solved are the main outcomes of this doctoral research work.
They have led to four journal publications \cite{Nagel:JAIS2015,Nagel:PEM2016,Nagel:JRUES2016,Nagel:JCP2016},
an equal number of conference papers \cite{Nagel:IPW2013:Proc,Nagel:SciTech2014:Proc,Nagel:ICVRAM2014:Proc,Nagel:ICASP2015:Proc}
and five other presentations \cite{Nagel:MascotNum2014:Pres,Nagel:PANACM2015:Pres,Nagel:SRES2015:Pres,Nagel:SIAMUQ2016:Pres,Nagel:ECCOMAS2016:Pres}.
The most important contributions \cite{Nagel:JAIS2015,Nagel:PEM2016,Nagel:JRUES2016,Nagel:JCP2016,Nagel:ICASP2015:Proc} are contained as individual chapters later on in the dissertation.
Postprints of the finally accepted and already published articles \cite{Nagel:JAIS2015,Nagel:PEM2016,Nagel:JRUES2016,Nagel:JCP2016}
after scholarly peer review and before the copyediting and typesetting are provided.
The conference paper \cite{Nagel:ICASP2015:Proc} is supplemented with three additional graphics whose inclusion was originally prevented by the template and page limit.
% BIBLIOGRAPHY
\printbibliography[heading=subsubbibliography,title={Journal papers},type=article]
\printbibliography[heading=subsubbibliography,title={Conference proceedings},type=incollection]
\printbibliography[heading=subsubbibliography,title={Other presentations},type=inproceedings]

\section{Outline}
% PARTS
An overview of the doctoral thesis and its structure is now given.
The document is basically divided into three parts.
It starts with basic introductions to uncertainty quantification and Bayesian inference in \cref{sec:Introduction}.
The key contributions of the research work in form of the most important publications are compiled in \cref{sec:Published}.
Some further unpublished investigations as well as a detailed hydrological case study are conducted in \cref{sec:Further}.
% CHAPTERS
A short overview of how the main topics and the associated publications are organized in parts and chapters is tabulated below.
Detailed chapter summaries follow directly thereafter.
% TABLE: THESIS OUTLINE
\begin{table}[htbp]
  \centering
  \begin{tabular}{llll}
    \toprule
    \multirow{2}{*}{\cref{sec:Introduction}} & \cref{sec:Uncertainty} & Uncertainty quantification     & \\
                                             & \cref{sec:Bayesian}    & Bayesian inference             & \\
    \midrule
    \multirow{5}{*}{\cref{sec:Published}}    & \cref{sec:PEM}         & Aleatory variability           & \cite{Nagel:PEM2016} \\
                                             & \cref{sec:JRUES}       & Hamiltonian Monte Carlo        & \cite{Nagel:JRUES2016} \\
                                             & \cref{sec:JAIS}        & NASA Langley challenge         & \cite{Nagel:JAIS2015} \\
                                             & \cref{sec:ICASP}       & Structural masonry             & \cite{Nagel:ICASP2015:Proc} \\
                                             & \cref{sec:JCP}         & Spectral Bayesian inference    & \cite{Nagel:JCP2016} \\
    \midrule
    \multirow{2}{*}{\cref{sec:Further}}      & \cref{sec:Transport}   & Optimal transportation         & \\
                                             & \cref{sec:Hydrology}   & Hydrological model calibration & \\
    \bottomrule
  \end{tabular}
\end{table}

\subsection{Elementary introductions}
% BASIC INTRODUCTIONS
The thesis starts with two introductory chapters on uncertainty quantification and Bayesian inference in engineering problems.
This material provides the necessary background information as well as complementary perspectives on the more advanced developments that follow.
Forward and inverse problems are discussed within the framework of probabilistic uncertainty quantification.
State-of-the-art techniques for the computational forward and backward propagation of uncertainty are reviewed.
\par % UNCERTAINTY QUANTIFICATION
An introduction to uncertainty quantification with a clear focus on probabilistic methods and forward propagation is provided in \cref{sec:Uncertainty}.
The quantitative characterization of the response distribution of a mechanical model due to randomness in the input parameters,
e.g.\ material properties, object geometry, environmental loads or operating conditions, is the classical example problem.
Monte Carlo simulation, Taylor series expansions and more global metamodeling techniques are presented in this context.
The latter includes stochastic spectral methods such as polynomial chaos expansions which are used throughout the whole dissertation.
Non-intrusive computations based on a linear least squares minimization problem and its ordinary least squares solution are concentrated on.
The curse of dimensionality as well as the hope for sparsity are discussed.
\par % BAYESIAN INFERENCE
\cref{sec:Bayesian} contains an elementary introduction to the Bayesian data analysis of engineering systems.
This offers a principled way of quantifying and reducing epistemic parameter uncertainties.
Experimental data that are only indirectly associated to the actual quantities of interest are analyzed to that end.
An example is the determination of actually uncertain properties of a material with measurements of its behavior under certain test conditions.
Basic inferential principles founded on the likelihood function as well as the prior and the posterior distribution are introduced.
More advanced topics such as evidence-based model comparison and some practical issues related to the parametrization of statistical models are also covered.
Conventional approaches to computational Bayesian inference based on random sampling or mathematical optimization are surveyed, e.g.\ Markov chain Monte Carlo and variational inference.
The convenient calculation of the extremely large or small quantities that typically arise in Bayesian computations is discussed.
Bayesian inverse problems are dealt with in greater detail together with related issues such as the quantification of model prediction error.

\subsection{Aleatory variability}
% ALEATORY VARIABILITY
While the Bayesian solution to inverse problems satisfactorily accounts for epistemic types of uncertainty, it does not allow for the incorporation of aleatory types.
Two examples of this form of uncertainty are the stochastic variation of the environmental or operating conditions over time
and the randomness within an ensemble of structural elements due to manufacturing tolerances.
This is a major limitation and motivates the research question of how to deal with aleatory input variability in Bayesian inverse problems.
The answer to the question is a core topic in the dissertation and occupies two chapters at the very least.
\par % PEM: ALEATORY VARIABILITY IN INVERSE PROBLEMS
A hierarchical framework for managing heterogeneous types of uncertainty in Bayesian inverse problems is proposed in \cref{sec:PEM}.
The formulation rests on multilevel models that interrelate different system components through deterministic simulators and conditional probability distributions.
It allows one to reduce the epistemic uncertainty of unknowns that are fixed yet unknown and to identify the distribution of quantities that vary throughout a series of experiments.
Random measurement noise and aleatory nuisance variables are taken into account at the same time.
All available sources of information such as experimental data and expert knowledge can be harnessed and optimally combined.
This is especially important in civil engineering applications where information is scarce and uncertainty dominates.
The framework is demonstrated and its computational challenges are identified through estimating the material variability across equally manufactured structural elements.
Inference can be either based on a low-dimensional formulation with an integrated likelihood function or on a high-dimensional variant with many unknowns.
\par % JRUES: HAMILTONIAN MONTE CARLO
After the framework for Bayesian inversion under epistemic and aleatory uncertainty has been elaborated, specialized solvers have to be implemented for computing the posterior efficiently.
In \cref{sec:JRUES} we propose Hamiltonian Monte Carlo in order to cope with the practical difficulties of high-dimensional Bayesian multilevel modeling.
As a member from the extended Markov chain Monte Carlo family, the algorithm explores the posterior in a sampling-based manner.
The idea is to embed the space of the unknown parameters in an auxiliary space and to perform the Markovian updates in such a way that they mix well in the original space of interest.
This principle is inspired by systems and concepts from classical and statistical mechanics, i.e.\ Hamiltonian dynamics and the Boltzmann distribution.
It calls for derivatives of the log--posterior density and the forward model.
Hamiltonian Monte Carlo is shown to be a highly efficient solver for inverse problems under uncertainty and variability.
It drastically outperforms a random walk Metropolis algorithm in a benchmark problem with more than hundred unknowns.
The posterior can be sampled almost independently.

\subsection{Computational methods}
% COMPUTATIONAL METHODS
Beyond the treatment of aleatory variability in inverse problems, the development of novel methods for computational Bayesian inference is another central theme of the thesis.
Although Hamiltonian Monte Carlo is an attractive algorithm, it does not overcome the principal limitations of sampling techniques in general.
Fundamental alternatives to Markov chain and sequential Monte Carlo are thus needed.
Two entirely different approaches to compute the posterior distribution numerically are developed and investigated.
They are based on approximations of the posterior probability density function or correspondingly distributed random variables.
The identification of the thermal properties of a composite material with inclusions poses a comparably simple inverse heat conduction problem that serves testing and demonstration purposes.
\par % JCP: SPECTRAL LIKELIHOOD EXPANSIONS
Spectral Bayesian inference is developed in \cref{sec:JCP} as a completely new and pretty elegant technique for posterior computations.
The main idea is to decompose the likelihood function into a converging series of orthogonal polynomials.
If orthogonality is defined with respect to the prior weight, this spectral likelihood expansion has some surprisingly interesting properties.
It gives rise to a nonparametric representation of the normalized posterior density and enables semi-analytic and sampling-free inference.
The model evidence as well as the posterior moments are related to the expansion coefficients from which they can be easily extracted.
Posterior uncertainty propagation through general computational models can be accomplished based on prior polynomial chaos expansions.
It is proposed to compute the expansion coefficients by a discrete linear least squares projection.
A perturbation-theoretic interpretation of the orthogonal series expansion of the posterior suggests a change of the reference density from the prior to an auxiliary weight function.
This improves the accuracy and efficiency of the spectral method dramatically.
The advantages and shortcomings of spectral Bayesian inference are highlighted by reference to classical distribution fitting and the inverse heat conduction problem.
\par % ADDITIONAL CHAPTER: TRANSFORMATION-BASED INFERENCE
Another approach to computational Bayesian inference that was recently devised by Prof.\ Youssef Marzouk and his group at MIT is investigated in \cref{sec:Transport}.
It is based on the translocation of probability mass from the prior to the posterior measure.
A function of random variables distributed according to the prior is constructed in such a way that the transformed variables follow the posterior.
One can establish a connection between variational Bayesian inference and this transport-based formulation.
This permits to compute the posterior by solving an optimization problem with an information-theoretic optimality criterion.
The random variable transformation is parametrized through multivariate polynomials up to a certain degree.
After the computation of a suitable transform, one can draw independent and equally weighted samples from the posterior.
This compelling feature distinguishes the approach from conventional sampling techniques.

\subsection{Practical applications}
% PRACTICAL APPLICATIONS
A number of inverse problems involving real data and forward models are solved with the previously developed methods towards the end of the thesis.
This can be seen as a justification and appreciation of the more formal developments, but should not hide the fact that it actually motivated some of them in the first place.
Spectral Bayesian inference for instance originated in the context of the NASA Langley multidisciplinary uncertainty quantification challenge.
The initial intention to use a polynomial approximation of the log--likelihood function in conjunction with Markov chain Monte Carlo sampling
has evolved into the idea for a spectral likelihood expansion which renders further posterior sampling completely unnecessary.
\par % JAIS: PERFECT DATA & NASA UQ CHALLENGE
\cref{sec:JAIS} is the outcome of participating in the NASA Langley uncertainty quantification challenge in 2013--2014.
The challenge contained a set of interlinked uncertainty quantification problems from the domain of aerospace engineering.
In this chapter the calibration sub-problem is interpreted and solved in the developed framework of Bayesian multilevel modeling.
A black-box model describing the behavior of a miniature civilian aircraft under adverse flight conditions and associated data were provided by NASA.
The primary goal was the reduction of epistemic uncertainty of the model parameters that are fixed yet unknown
and the identification of the hyperparameters that determine the distribution of the aleatory variables.
Due to some peculiarities of the problem statement related to a zero-noise or perfect data condition,
the likelihood function only arises as the solution to a secondary uncertainty forward propagation problem.
For that reason it cannot be evaluated exactly.
A statistical approximation of the likelihood based on Monte Carlo simulation and kernel density estimation is therefore proposed.
Employing this biased and noisy likelihood estimator for sampling the posterior via Markov chain Monte Carlo alters the Metropolis--Hastings transition kernel.
The induced modifications on the posterior level are investigated and mitigated by means of partial data augmentation.
\par % ICASP: STRUCTURAL MASONRY
Bayesian multilevel modeling also facilitates problem-solving in structural masonry.
A hierarchical approach to assess the compressive strength of masonry walls is presented in \cref{sec:ICASP}.
Many current methods suffer from their homogeneous treatment of the composite material or simply fail in uncertainty quantification.
Other standardized methods overpredict the compressive strength to an alarming extent.
The devised approach allows one to improve the accuracy of the predictions and assess their quality.
It models structural masonry heterogeneously and quantifies the arising uncertainties consistently.
System-level data related to the masonry wall specimens and component-level data of the brick units and mortar are analyzed jointly.
The experimental data were collected in a series of compressive tests performed by Dr.\ Nebojsa Mojsilovic and his students
in the laboratories of the Institute of Structural Engineering (IBK) at the ETH Z\"{u}rich.
After the calibration of the unknown parameters and hyperparameters, one can probabilistically predict the compressive strength based on measurements of the constituent ensembles used.
\par % ADDITIONAL CHAPTER: HYDROLOGICAL ENGINEERING
Another real-world problem is solved in \cref{sec:Hydrology} where a hydrological urban drainage simulator is calibrated.
Epistemic parameter uncertainties are reduced while random measurement noise and systematic modeling errors are anticipated and statistically identified.
This allows for a thorough treatment of the emerging sources of error and uncertainty in the dynamical simulation of water systems.
It also suggests the possibility for model correction.
The catchment area of Adliswil, a municipality located around the river Sihl at the southern end of the city Z\"{u}rich, is studied during a rainfall event.
Experimental data and training runs of the simulator were provided by the Swiss Federal Institute of Aquatic Science and Technology (Eawag) in D\"{u}bendorf.
Advanced techniques for dimension reduction, surrogate modeling and stochastic sampling are combined to this effect.
Principal component analysis allows us to reduce the output dimensionality of the deterministic simulator that predicts a whole times series.
Sparse polynomial chaos expansions are subsequently used in order to emulate the input-output relationship the forward model defines.
The posterior distributions of two different Bayesian models are sampled via Markov chain Monte Carlo techniques and compared with each other.

\end{refsection}