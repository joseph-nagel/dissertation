% CHALLENGE & REWARD
Addressing the uncertainty characterization subproblem of the NASA Langley multidisciplinary UQ challenge has turned out to be a challenging yet rewarding task.
% MULTILEVEL FRAMEWORK
We began with formulating a generic Bayesian multilevel framework for managing different types of forward model input uncertainties in complex inverse problems.
% ADDITIONAL POTENTIAL
Incidentally this showed how the problem could be solved for ``imperfect'' data, e.g.\ in the presence of additional measurement noise, 
and how the entirety of problem unknowns, including those that are not of declared inferential interest, can be deduced.
% SELF-REFERENCE
Although these were not the guiding questions, this is a future research direction in its own \cite{Nagel:PEM2016}.
% PERFECT DATA
Bayesian structural prior modeling served as a foundation for devising a multilevel model in the zero-noise or ``perfect'' data limit,
i.e.\ the data space has been endowed with a probability model that was based on uncertainty propagation.
% PROBLEM FORMULATION
Ensuing from those general considerations we have interpreted and solved the challenge problem as Bayesian calibration of a suitably defined multilevel model.
% INTERPRETATIONS & ASSUMPTIONS
We thoroughly commented on the assumptions that the adopted approach rests upon as well as the interpretations it entails.
\par % POSTERIOR FIDELITY & OPEN QUESTIONS
In turn the problem solution has given rise to new questions of theoretical and practical relevance alike.
Posterior fidelity was discussed in the context of MCMC posterior exploration and online uncertainty propagation.
First related thoughts were given and an in-depth consideration has been initiated.
% STARTING POINT
The starting point of the latter could be \cref{eq:JAIS:PerfectData:MSE,eq:JAIS:PerfectData:KernelSmoothedLikelihood,eq:JAIS:MCMC:MHCorrection,eq:JAIS:MCMC:LikelihoodRatio}.
% PARTIAL DATA AUGMENTATION
With the objective of improving the fidelity of the final results we demonstrated how one can exploit partial data augmentation.
In addition to improving the estimation of the QoI, in principle this approach allows to infer such problem unknowns that inferential interest is not particularly focused on.
That way partial data augmentation has provided further insight into the calibration problem posed.
% OUR CONTRIBUTION
In sum we hope that these efforts prove to be a solid contribution to the NASA challenge problem in particular and to the theory and practice of Bayesian data analysis and uncertainty quantification in general.
% FUTURE RESEARCH
Future research work encompasses the design of more sophisticated methods to simulate the likelihood function and the rigorous assessment of the posterior fidelity.