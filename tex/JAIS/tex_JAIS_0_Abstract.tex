% FRAMEWORK
A probabilistic framework for Bayesian inference and uncertainty analysis is developed.
It allows to address inverse problems in experimental situations where data is scarce and uncertainty is ubiquitous.
% OBJECTIVE
The uncertainty characterization subproblem of the NASA Langley multidisciplinary uncertainty quantification challenge serves as the motivating application example.
From responses of a computational model the goal is to learn about unknown model inputs that are subject to multiple types of uncertainty.
This objective is interpreted and solved as Bayesian multilevel model calibration.
% ZERO-NOISE
The zero-noise or ``perfect'' data limit is investigated.
Thereby the likelihood function is defined as a solution to forward uncertainty propagation.
% TECHNIQUES
Posterior explorations are based on suitable Markov chain Monte Carlo algorithms and stochastic likelihood simulations.
% FIDELITY
An unforeseen finding in this context is that the posterior distribution can only be sampled with a certain degree of fidelity.
% DATA AUGMENTATION
Partial data augmentation is introduced as a means to improve the error statistics of likelihood estimations and the fidelity of posterior computations.