% UQ CHALLENGE
The NASA Langley multidisciplinary uncertainty quantification challenge has raised contemporary open questions to uncertainty quantification (UQ) \cite{NASA:Crespo2014:Proc,NASA:Crespo2013:FAQ:URL}.
Altogether it consists of five subproblems that deal with uncertainty characterization, sensitivity analysis, uncertainty propagation, extreme case analysis and robust design.
These problems originate from a specific aerospace application which is part of greater efforts to reduce the rate of fatal loss-of-control accidents \cite{NASA:Foster2005:Proc,NASA:Jordan2008:Proc,NASA:Crespo2012}.
An abstract and widely discipline-independent problem formulation prompts researchers and practitioners from various fields in academia and industry to devise generic solutions to the problems.
% SOME BACKGROUND
A dynamically scaled, free-flight model of a remotely piloted, twin-turbine powered transport aircraft is the physical system under consideration.
It serves as a prototyping and experimentation testbed for flight control in adverse situations, e.g.\ under structural damage or component failure.
Parameter uncertainties of this subscaled model reflect the uncertainties in aerodynamic conditions and losses in control effectiveness.
\par % SUBPROBLEM A
In this contribution we address the uncertainty characterization subproblem of the challenge posed.
With given responses of a computational model the challenge is to learn about the unknown inputs that parametrize the flying conditions.
Throughout the experiments data are collected while model inputs are subject to epistemic uncertainty and aleatory variability \cite{Uncertainty:Faber2005,Uncertainty:Kiureghian2009}.
% (HYPER)PARAMETERS
Inference therefore focuses on physically fixed model parameters as well as on so-called hyperparameters.
The latter determine the distribution of such model inputs that are variable during experimentation.
% BAYESIAN PARADIGM
We approach the problem from a Bayesian perspective to statistical inversion and uncertainty quantification \cite{Bayesian:Hadidi2008,Bayesian:Beck2010,Bayesian:Stuart2010}.
% CLASSICAL INVERSION % MULTILEVEL MODELING
While classical Bayesian inversion allows to estimate constant model parameters, the additional identification of hyperparameters requires hierarchical modeling approaches.
Hierarchical models were mainly developed in biological statistics \cite{Multilevel:Davidian2003,Multilevel:Banks2012}
and they are only slowly being adopted within the engineering community \cite{Multilevel:Rocquigny2009,Multilevel:Celeux2010,Multilevel:Barbillon2011}.
\par % MOTIVATION: FRAMEWORK
The goal of this paper is the development of a framework along with computational tools for attacking inverse problems under aleatory and epistemic parameter uncertainty.
We combine classical inversion and hierarchical modeling into a Bayesian multilevel framework that allows to tackle the general class of problems that the uncertainty characterization subproblem typifies.
% BAYESIAN MULTILEVEL MODELING
Within a probabilistic setting this eventually allows for an elegant formulation and efficient numerical solution.
% PERFECT DATA
The foundations of inverse modeling in conjunction with ``perfect'' data, i.e.\ in the zero-noise limit, and parameter uncertainty are laid.
Randomness in the data is then solely attributed to a probability model of the input arguments of a computational ``blackbox'' solver.
% LIKELIHOOD FUNCTION
The likelihood is formulated as a solution to uncertainty propagation.
Since this renders its evaluation analytically intractable, statistical estimators based on the Monte Carlo method and kernel density estimation are proposed.
% POSTERIOR DISTRIBUTION
In this context the induced type of posterior approximation is investigated.
% HEURISTIC CONSIDERATIONS
Heuristic ways of tuning free algorithmic parameters, e.g.\ the kernel bandwidth, are presented.
% PARTIAL DATA AUGMENTATION
Partial data augmentation is proposed in order to improve likelihood estimations through automatic kernel bandwidth selection and to enhance the fidelity of the posterior.
\par % ORGANIZATION
The manuscript is organized as follows.
In \cref{sec:JAIS:Multilevel} a generic Bayesian multilevel framework for inversion under uncertainty will be initially formulated.
For the solution of the NASA Langley UQ challenge we will devise a statistical model involving ``perfect'' data in \cref{sec:JAIS:PerfectData}.
Computational key challenges posed by Bayesian inference in the present context will be discussed in \cref{sec:JAIS:Computation}.
The challenge problem will be cast as multilevel inversion in \cref{sec:JAIS:Challenge} and in the subsequent \cref{sec:JAIS:Analysis} our results will be presented.
In \cref{sec:JAIS:Augmentation} data augmentation will be utilized in order to ensure a sufficient degree of algorithmic efficiency and posterior fidelity.
We will conclude in \cref{sec:JAIS:Conclusion} where the gathered experience from solving the NASA UQ challenge problem is summarized.