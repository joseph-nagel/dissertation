% EUROCODE
In \cite{Standard:Eurocode6:1-1} it is tried to relate the compressive strength of masonry to the resistances of its brick and mortar components.
The relationship is realized as a power function
\begin{equation} \label{eq:ICASP:EmpiricalRelation}
  f_w = k^\prime f_b^{\alpha^\prime} f_m^{\beta^\prime}.
\end{equation} 
On the one hand, the compressive strength of masonry is summarized by the characteristic value \(f_w\), i.e.\ a \(\unit[5]{\%}\)-quantile.
On the other hand, \(f_b\) denotes the normalized mean compressive strength of the units and \(f_m\) denotes the mean compressive strength of the mortar. 
Estimates of the constants \((k^\prime,\alpha^\prime,\beta^\prime)\) are given for different types of masonry.
% JCSS
In \cite{Standard:JCSSProbabilisticCode} the empirical relation \cref{eq:ICASP:EmpiricalRelation} is interpreted similarly.
Here \(f_w\), \(f_b\) and \(f_m\) represent the mean values of the corresponding distributions.
Different prior estimates of the coefficients \((k^\prime,\alpha^\prime,\beta^\prime)\) are provided.
% ALPHA + BETA = 1
The coefficients are often set so that they (approximately) satisfy \(\alpha^\prime + \beta^\prime = 1\).
This choice can be justified for reasons of the physical dimension in \cref{eq:ICASP:EmpiricalRelation}.
\par % EXTENSIONS
Ensuing from these semi-probabilistic models, a variety of extensions have been proposed in the literature.
% PROBABILISTIC EXTENSIONS
There are probabilistic reinterpretations of \cref{eq:ICASP:EmpiricalRelation} based on lognormal distributions \cite{Masonry:Schueremans2006,Masonry:Sykora2010,Masonry:Sykora2014:a,Masonry:Sykora2014:b}.
In other studies the model uncertainty of \cref{eq:ICASP:EmpiricalRelation} is quantified \cite{Masonry:Dymiotis2002,Masonry:Glowienka2006}.
A conjugate Bayesian updating approach based on Gaussian distributions is presented in \cite{Masonry:Mojsilovic2009}.
Another idea is to establish a connection between the compressive strengths of masonry and its components via artificial neural networks \cite{Masonry:GarzonRoca2013}.
\par % CRITICISM
These previous approaches suffer from the fact that they either do not clearly distinguish between epistemic and aleatory shares of uncertainty or they neglect material heterogeneity.
Fitting the parameters of a probabilistic extension of \cref{eq:ICASP:EmpiricalRelation} is a problem that has hardly been satisfactorily solved as yet.