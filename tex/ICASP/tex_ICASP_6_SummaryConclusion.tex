% FUNDAMENTAL ACHIEVEMENT
It was demonstrated how hierarchical Bayesian models can serve the purpose of assessing the compressive strength of structural masonry.
This establishes a fully probabilistic alternative to the existing semi-probabilistic approaches.
The hierarchical framework offers versatile and powerful tools of uncertainty quantification and information aggregation at multiple system levels.
Different types of uncertainty, i.e.\ ignorance and variability, are thoroughly managed, while heterogeneous types of information, e.g.\ data and expert knowledge, are consistently utilized.
This way the analysis of the masonry wall resistance can be based on large-scale compression tests as well as on inexpensive tests of brick unit and mortar samples.
\par % SUGGESTIONS
Our hope is that this possibility will encourage experimenters in entirely publishing their collected data.
In fact it seems to be commonplace to quote statistical data summaries only, e.g.\ sample means or characteristic values.
The proposed methodology, however, allows to process the acquired data as a whole.
\par % PARAMETRIZATION
A number of questions have arisen.
It is queried if \cref{eq:ICASP:AleatoryUncertainty} is an adequate representation of the distribution of masonry compressive strength in terms of distributional parameters of the components.
With regard to the complexity of structural masonry, its failure modes and their dependency on the quality of workmanship,
the relations \cref{eq:ICASP:EmpiricalRelation,eq:ICASP:ProbabilisticInterpretation} are oversimplifying.
They were inspired by the structure of current models but lack a solid physical foundation.
For future studies this motivates the introduction of model uncertainty in addition to the emerging parameters uncertainties.
Beyond that future work will also involve the construction and objective selection of better system-level models of aleatory variability.
A more fundamental question concerns the general suitability of empirical relations for any probabilistic extension whatsoever.
% DISCREPANCY
Another raised issue relates to the observed mismatch between measurements and code-predictions.
We were not able to explain this discrepancy.