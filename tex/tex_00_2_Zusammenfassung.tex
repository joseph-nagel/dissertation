% BAYESSCHE INFERENZ
Bayessche Inferenz erlaubt die Miteinbeziehung heterogener Informationen und die Reduktion epistemischer Unsicherheiten beim Lösen inverser Probleme.
Physikalische Modelle, Expertenwissen und Messdaten werden statistisch ausgewertet, um auf die unbekannten Modellparameter zurückzuschließen.
Die Prior- und die Posterior-Wahrscheinlichkeitsverteilung repräsentieren die Unsicherheit der Unbekannten vor und nach der Analyse.
Der Übergang vom Prior in den bedingten Posterior erfolgt nach dem Lehrsatz von Bayes und spiegelt den erzielten Informationsgewinn wieder.
\par % POSTERIOR-BERECHNUNG
Das Charakterisieren der Posterior-Verteilung stellt die größte Herausforderung der Bayesschen Datenanalyse dar.
Weil nur denkbar einfache Probleme analytische Lösungen besitzen, muss man den Posterior meistens numerisch berechnen.
Markov-Ketten-Monte-Carlo-Verfahren werden zur Bewältigung dieser schwierigen Aufgabe eingesetzt.
Die große Anzahl der dafür erforderlichen Vorwärtsmodellläufe verhindert die rechnergestützte Inferenz in vielen Anwendungsgebieten.
Dies gilt insbesondere im Bauingenieurwesen und Maschinenbau sowie in der Luft- und Raumfahrttechnik.
\par % HAUPTBEITRÄGE
Ziel dieser Doktorarbeit ist es, neue Methoden zur Bayesschen Wahrscheinlichkeitsanalyse in komplexen und realistischen Ingenieuranwendungen zu entwickeln.
Ein einheitliches Rahmenkonzept für inverse Probleme unter epistemischer Unsicherheit und aleatorischer Variabilität wird zu diesem Zweck ausgearbeitet.
Um den damit verbundenen Rechenaufwand zu verringern, wird ein effizienter Hamiltonscher Monte-Carlo-Algorithmus verwendet.
Darüber hinaus werden völlig neuartige Ansätze zur Berechnung der Posterior-Wahrscheinlichkeitsverteilung vorgestellt und untersucht.