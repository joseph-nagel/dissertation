% THREE APPROACHES
Simple updating, staged estimation and multilevel inversion were proposed as Bayesian approaches to uncertainty reduction and information gathering in hierarchically defined inversed problems.
Multilevel inversion was shown to outclass simple updating and staged estimation in terms of consistency and effectiveness.
Specifically it facilitates the coherent inference of the problem unknowns where available prior information and experimental data are optimally combined.
In the case that intermediate model variables are of inferential interest, this allows for pooling statistical strength.
% BORROWING STRENGTH
The potential of borrowing strength was demonstrated for the optimal estimation of material properties.
The mutual exchange of information between specimens from a statistical ensemble was investigated.
Ensemble members that are subject to a high degree of uncertainty can be advantageously assessed by exploiting information from members that are more strongly informed by the data.
In our example application this allowed to mitigate the influence of high measurement uncertainty.
\par % HMC PERFORMANCE
Moreover posterior exploration in high-dimensional parameter spaces was addressed.
HMC was proven to be a practical and highly efficient sampler for hierarchical problems.
For the system under consideration it outperforms RWM by two orders of magnitude as measured by the number of effective posterior samples that can be simulated within a given execution time. 
Put another way, for achieving a certain number of effective draws it reduces the execution time by two orders of magnitude.
% MORE COMPLEX MODELS 
That way the HMC algorithm enables uncertainty quantification in more complex problems where the employment of classical MCMC techniques would be unfeasible.
The high computational cost associated with traditional algorithms may easily exceed the available budget
for problems that involve more sophisticated representations of uncertainty or more resource intensive forward models.
\par % STOCHASTIC MODELS
With increasing dimensionality of the parameter space, e.g.\ due to a refined uncertainty model, HMC promises to yield relative speedups that are even higher than the one observed in our benchmark application.
% DETERMINISTIC MODELS
For more advanced forward models, efficient and sufficiently accurate means to evaluate their derivatives have to be devised.
A promising idea is the use of surrogate models such as polynomial chaos expansions.
Research in this direction is in progress.