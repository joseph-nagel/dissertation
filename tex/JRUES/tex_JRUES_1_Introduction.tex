% UNCERTAINTY QUANTIFICATION
Bayesian inference establishes a flexible framework for solving inverse problems.
Given measured responses of a forward model, it allows for reducing epistemic uncertainty of unknown parameters \cite{Bayesian:Hadidi2008,Bayesian:Beck2010} and,
within the context of hierarchical modeling, for quantifying the aleatory uncertainty of unobservables \cite{Multilevel:Davidian2003,Multilevel:Banks2012}.
% ALEATORY VARIABILITY
The latter type of problem is often encountered in social science or biological statistics.
Yet it is of great interest for engineering applications, too.
It allows one to study the natural variability of physical parameters that cannot be directly measured.
This involves the variability of material properties as a result of the unavoidable uncertainties in the manufacturing process
or due to spatial and temporal changes in the environmental conditions \cite{Uncertainty:Faber2005,Uncertainty:Kiureghian2009}.
\par % LITERATURE
In the recent literature a number of approaches have been devised that aim at fitting the hyperparameters of the aleatory distribution of forward model inputs
\cite{Multilevel:Rocquigny2009,Multilevel:Celeux2010,Multilevel:Barbillon2011}.
These approaches build upon the marginalization of varying inputs at the level of the likelihood function. 
This way of proceeding typically leads to low-dimensional estimation problems where the major difficulty lies in computing the integrated likelihood.
The joint inference of the distributional hyperparameters and the experiment-specific realizations of the variables constitutes a higher-dimensional problem \cite{Multilevel:Fu2015,Multilevel:Behmanesh2015,Nagel:PEM2016}.
Even though the marginalized and the joint problem variants are equivalent with regard to hyperparameter calibration,
the former formulation does not allow for the estimation of realizations of the variable parameters.
To this effect one has to rely on the joint problem formulation.
\par % JOINT FORMULATION
In this paper the joint parameter/hyperparameter inference is studied in view of reducing the uncertainty in the parameters.
The goal is to demonstrate the advantages of this formulation and to overcome its computational difficulties.
% BORROWING STRENGTH
First, we investigate borrowing strength \cite{Multilevel:Draper1992} as a means of information aggregation in inverse problems.
This statistical mechanism allows for an optimal estimation of individual material properties within a specimen ensemble.
We prove that this is valuable in experimental situations where uncertainty is dominant and information is heterogeneous.
Indeed these are characteristics of problems in civil engineering.
% ENGINEERING PROBLEM
Specifically the system under consideration is an ensemble of identically manufactured beams that are individually tested in a series of experiments.
A situation is considered where evidence is unevenly distributed throughout the ensemble members,
i.e.\ the properties of some members can be measured with high accuracy whereas others are poorly informed by the data.
% HAMILTONIAN MONTE CARLO
Second, in order to alleviate problems of Markov chain Monte Carlo (MCMC) \cite{MCMC:Robert2004,MCMC:Brooks2011} for posterior exploration in high-dimensional parameter spaces, we propose Hamiltonian Monte Carlo (HMC) \cite{MCMC:Duane1987}.
The latter is an advanced MCMC technique that allows for gradient-assisted posterior computation with auxiliary variables.
We show that HMC is ideally suited and extremely efficient for borrowing strength in inverse problems.
\par % STRUCTURE
The main part of the paper is structured in the following way.
Stochastic inversion and multilevel modeling are reviewed first.
On this basis, borrowing strength and information accumulation are investigated.
An introduction to HMC sampling is provided after that.
In order to study the proposed methods, a numerical experiment with simulated data is conducted.
Concluding remarks are given in the end.