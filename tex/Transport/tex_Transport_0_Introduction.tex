% INTRODUCTION
This chapter provides an introduction to and demonstration of Bayesian inference via transport maps.
Bayesian updating is first recast as a random variable transformation and then solved as an optimization problem.
The latter involves an information-theoretic optimality criterion.
In particular, the relative entropy of the back-transformed posterior from the prior is minimized.
After posing the problem that way, it becomes regularized in the framework of optimal transportation theory.
\par % YOUSSEF MARZOUK
The transport map--based formulation was originally introduced in \cite{Mapping:ElMoselhy2012}.
A nice overview of the optimal transportation of probability measures for purposes of Bayesian inference is given in \cite{Mapping:Marzouk2016}.
% SEBASTIAN REICH
Similar ideas had also emerged in the context of sequential data assimilation, an overview of which can be found in \cite{Mapping:Reich2013:b,Bayesian:VanLeeuwen2015}.
% BAYESIAN COMPUTING
Transformation-based inference joins the ranks of the Bayesian methods reviewed in \cref{sec:Bayesian:BayesianComputations}.
It can be seen as a special case of variational Bayesian inference in \cref{sec:Bayesian:BayesianComputations:Variational},
where certain prior transformations constitute the parametric family of candidate distributions.
Beyond that, it shares commonalities with spectral Bayesian inference as presented in \cref{sec:JCP}.
\par % BAYESIAN IHCP
A simple inverse heat conduction problem is used for demonstration purposes.
Unknown thermal conductivities of a composite material are indirectly inferred from measurements of the temperature that are taken close to the boundary.
The prior is transformed into the corresponding posterior distribution.
Traditional Markov chain Monte Carlo sampling serves as the reference solution.
% INTERNATIONAL SYMPOSIUM (SRES)
Parts of this chapter were also presented at the International Symposium on Reliability of Engineering Systems
that was held on October 15--17, 2015 in Hangzhou, China \cite{Nagel:SRES2015:Pres}.
\par % OUTLOOK
The tutorial on Bayesian inference as a random variable transformation is structured as follows.
In \cref{sec:Transport:PriorTransformations} the optimal transportation from the prior to the posterior is investigated.
In \cref{sec:Transport:VariationalFormulation} a variational problem is formulated that allows for a numerical computation of the posterior.
Subsequently, \cref{sec:Transport:PracticalComputation} covers practical issues such as the parametrization of the map and the regularization of its computation.
In \cref{sec:Transport:ComparisonToSLEs} the pros and cons of the approach are weighed and compared to spectral Bayesian inference that was devised previously.
In \cref{sec:Transport:NumericalExperiment} an inverse heat conduction problem is solved as an illustrative example of transformation-based Bayesian inference.
Finally, \cref{sec:Transport:SummaryConclusion} contains some concluding remarks.