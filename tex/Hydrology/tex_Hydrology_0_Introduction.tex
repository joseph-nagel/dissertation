% HYDROLIGICAL MODEL CALIBRATION
This chapter deals with the Bayesian calibration of a dynamical urban drainage simulator.
The process-based simulation of water systems is typically expensive and yet highly uncertain \cite{Physics:Beven2012}.
That is why hydrological model calibration is an extremely important and difficult task.
% EMPLOYED TECHNIQUES
We use a combination of advanced methods in order to estimate the unknown model parameters and to quantify the measurement and prediction errors.
Principal component analysis is used for purposes of dimension reduction of the model outputs that constitute a whole times series.
The accordingly reduced outputs are then metamodeled as a function of the unknowns based on sparse polynomial chaos expansions.
Eventually the posterior distribution of the unknown parameters of the hydrological and the error model is sampled via Markov chain Monte Carlo.
\par % HYDROLIGICAL SIMULATOR
The process-oriented hydrological simulator predicts the outflow from a catchment area that receives rainfall.
It has been developed at the Swiss Federal Institute of Aquatic Science and Technology (Eawag), where it was used in the PhD dissertation of D.\ Machac \cite{Hydro:Machac2015:PhD}.
% AVAILABLE INFORMATION
By courtesy of Eawag we have access to the results of roughly two thousand training runs of the simulator for a single rainfall event.
Moreover, about six hundred measurements of the time-varying runoff at a single outlet during the event were made available.
% BLACK-BOX SITUATION
This describes a realistic black-box situation where the abovementioned hybridization of techniques for compression, metamodeling and calibration permits a synthesis of the supplied information.
\par % ERRORS AND UNCERTAINTIES
Measurement uncertainties and modeling errors are explicitly considered in two different Bayesian models.
Parametric uncertainties in the hydrological inputs are the main focus of both models.
They differ, however, in the degree of sophistication of how the inevitable deviation of the model predictions from the measurement data is represented.
The first simple model only acknowledges independent measurement noise, while the second model also accounts for random error correlation and systematic model discrepancy.
\par % OUTLINE
The present chapter is organized as follows.
A more detailed overview of the problem setup and the available information is provided in \cref{sec:Hydrology:ProblemSetup}.
The construction of the hydrological emulator is described in \cref{sec:Hydrology:Metamodeling}.
Bayesian parameter estimation and predictive model correction are performed in \cref{sec:Hydrology:BayesianCalibration}.
Finally it is summarized and concluded in \cref{sec:Hydrology:DiscussionConclusion}.