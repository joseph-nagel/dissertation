% SUMMARY OF THE RESULTS
Probabilistic calibration of a hydrological urban drainage simulator was accomplished in this chapter.
A combination of techniques for Bayesian and surrogate modeling was deployed for that purpose.
The original training runs of the simulator were statistically compressed and subsequently translated into a functional emulator.
Inference was then based on the exploration of the posterior distributions related to two different Bayesian models through Markov chain Monte Carlo.
\par % BAYESIAN MODELS
The two models differ in their ability to capture measurement and modeling uncertainties.
With the first model, that only acknowledges independently varying errors, the hydrological parameters were calibrated.
With the second Bayesian model, that also includes random error correlation and systematic model discrepancy,
we could additionally quantify the mismatch between the model predictions and the data throughout the rainfall event.
This was, however, accompanied by a lack of interpretability with regard to the corresponding estimates of the hydrological parameters.
\par % FORESHADOWING
This last observation was already foreshadowed in the very setup of the problem.
While the catchment area has multiple outlets, only a single one at the wastewater treatment plant was considered for parameter calibration purposes.
A coarse-grained parametrization of the unknowns based on crude averages over hundreds of sub-catchments and channels was used.
Moreover, the whole procedure was dependent on a single precipitation event for which the rainfall record was taken as if it were measured without error.
In this context one also has to mention that the predictions of the forward model were highly uncertain and only weakly sensitive to the calibration parameters.
\par % FUTURE EXTENSIONS
The preceding discussion suggests a number of possible extensions and future improvements that would allow for more complex and realistic modeling.
A more refined representation of the uncertain hydrological parameters could be based on a finer graining of the spatial resolution.
Sparse polynomial chaos expansions and advanced stochastic sampling schemes would allow one to cope with the associated increase in dimensionality.
Errors and uncertainties in the rainfall input data could be considered by treating and inferring the rainfall as additional unknowns.
If the data for various different precipitation events were available,
a more thorough representation and management of the encountered uncertainties could be based on hierarchical Bayesian modeling as developed in \cref{sec:PEM,sec:JRUES}.