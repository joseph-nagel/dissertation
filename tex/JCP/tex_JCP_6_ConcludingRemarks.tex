% SPECTRAL BAYESIAN INFERENCE
A spectral approach to Bayesian inference that focuses on the surrogate modeling of the posterior density was devised.
The likelihood was expanded in terms of polynomials that are orthogonal with respect to the prior weight.
Ensuing from this spectral likelihood expansion (SLE), the joint posterior density was expressed
as the prior that acts the reference density times a polynomial correction term.
The normalization factor of the posterior emerged as the zeroth SLE coefficient and
the posterior marginals were shown to be easily accessible through sub-expansions of the SLE.
Closed-form expressions for the first posterior moments in terms of the low-order spectral coefficients were given.
Posterior uncertainty propagation through general quantities of interest was established via a postprocessing of the higher-order coefficients.
The semi-analytic reformulation of Bayesian inference was founded on the theory and practice of metamodeling based on polynomial chaos expansions.
This allows one to compute the SLE coefficients by solving a linear least squares problem.
An analysis of the advantages and disadvantages of the proposed method eventually motivated a change of the reference density.
While the expansion of the posterior in terms of the prior may require substantial modifications,
its representation with respect to an auxiliary density many only require minor tweaks.
\par % NUMERICAL EXPERIMENTS
The possibilities and difficulties that arise from the problem formulation were exhaustively discussed and numerically demonstrated.
Fitting a parametric distribution to random data and identifying the thermal properties of a composite material served as benchmark problems.
These numerical experiments proved that spectral Bayesian inference works in principle and they provided insight into the mechanisms involved.
The convergence behavior of the SLE was studied based on the leave-one-out error.
It was found that high-degree SLEs are necessary in order to accurately represent the likelihood function and the joint posterior density,
whereas lower-order SLEs are sufficient in order to extract the low-level quantities of interest.
A change of the reference density allowed for reducing the order of the corrections required in order to represent the posterior with respect to the prior.
This helped in alleviating the curse of dimensionality to some extent.
\par % OPEN QUESTIONS
In turn, a number of follow-up questions were given rise to.
While the leave-one-out error performs well in quantifying the prediction errors of the SLE,
it turned out to be of limited use with regard to the errors of the corresponding posterior surrogate and its marginals.
A critical question thus relates to a means to assess the errors of these quantities and to diagnose their convergence.
This would assist in choosing experimental designs of a sufficient size.
Also, it would be desirable to quantify the estimation errors of individual expansion coefficients.
This would support the assessment of the efficiency and scalability of the approach
and the fair comparison with Monte Carlo, importance and Markov chain Monte Carlo sampling.
Another question is whether a constrained optimization problem can be formulated that naturally respects all prior restrictions.
This would remedy the potential problem of illegitimate values of the posterior moments.
In order to handle a broader spectrum of statistical problems, SLEs would have to be extended to dependent prior distributions and noisy likelihood functions.
For increasing the computational efficiency beyond the change of the reference density, it is conceivable to deploy advanced techniques from metamodeling and machine learning.
This includes piecewise polynomial models, expansions in a favorable basis and the use of sparsity-promoting regression techniques.
Yet another important issue concerns the practical applicability of the presented framework to problems with higher-dimensional parameter spaces.
In future research efforts we will try to address the abovementioned issues and to answer this principal question.