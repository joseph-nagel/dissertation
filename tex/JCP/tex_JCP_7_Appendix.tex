\section{Univariate polynomials} \label[subappsec]{sec:JCP:App:Polynomials}
% UNIVARIATE POLYNOMIALS
The main properties of two classical orthogonal families of polynomials were shortly summarized in \cref{tab:JCP:PCE:UnivariateFamilies},
i.e.\ the domain of definition, the associated weight function and the norm.
The first six members of these univariate Hermite polynomials \(\{H_\alpha\}_{\alpha \in \mathds{N}}\)
and Legendre polynomials \(\{P_\alpha\}_{\alpha \in \mathds{N}}\) are listed in \cref{tab:JCP:App:Polynomials}.
Higher order members can be defined via recursive or differential relations.
These polynomials can be used for the construction of the multivariate polynomial basis \(\{\basis_\alpha\}_{\alpha \in \mathds{N}}\) in \cref{eq:JCP:PCE:MultivariatePolynomial}.
Note that this orthonormal basis is normalized via \(\basis_\alpha = H_\alpha / \sqrt{\alpha !}\) or \(\basis_\alpha = P_\alpha / \sqrt{1/(2 \alpha + 1)}\).
% TABLE: LOW-ORDER POLYNOMIALS
\begin{table}[htbp]
  \caption[Low-order polynomials]{Low-order polynomials.}
  \label{tab:JCP:App:Polynomials}
  \centering
  \begin{tabular}{rll}
    \toprule
    \(\alpha\) & \(H_\alpha(x)\), \(x \in \mathds{R}\) & \(P_\alpha(x)\), \(x \in [-1,1]\) \\
    \midrule
    \(0\) & \(1\)                 & \(1\) \\
    \(1\) & \(x\)                 & \(x\) \\
    \(2\) & \(x^2 - 1\)           & \((3x^2 - 1) / 2\) \\
    \(3\) & \(x^3 - 3x\)          & \((5x^3 - 3x) / 2\) \\
    \(4\) & \(x^4 - 6x^2 + 3\)    & \((35x^4 - 30x^2 + 3) / 8\) \\
    \(5\) & \(x^5 - 10x^3 + 15x\) & \((63x^5 - 70x^3 + 15x) / 8\) \\
    \bottomrule
  \end{tabular}
\end{table}

\section{Low-order QoIs} \label[subappsec]{sec:JCP:App:QoIs}
% LOW-ORDER QoIs
The representation of six low-order QoIs in terms of the normalized Hermite and Legendre polynomials is given in \cref{tab:JCP:App:QoI} below.
Those expansions can be used in order to compute the first posterior moments, e.g.\ as shown in \cref{eq:JCP:SLE:PosteriorMargMean,eq:JCP:SLE:PosteriorMargVariance,eq:JCP:SLE:PosteriorCovariance}.
Note that the representations in the orthonormal bases directly follow from a change of basis and the substitutions
\(H_\alpha = \sqrt{\alpha !} \basis_\alpha\) and \(P_\alpha = \sqrt{1/(2 \alpha + 1)} \basis_\alpha\).
% TABLE: QoI-REPRESENTATIONS
\begin{table}[htbp]
  \caption[Low-order QoIs]{Low-order QoIs.}
  \label{tab:JCP:App:QoI}
  \centering
    \begin{tabular}{rll}
      \toprule
      QoI & Hermite expansion & Legendre expansion \\
      \midrule
      \(1\phantom{^{2}}\) & \(\basis_0\)                                               & \(\basis_0\) \\
      \(x\phantom{^{2}}\) & \(\basis_1\)                                               & \(\basis_1 / \sqrt{3}\) \\
      \(x^2\)             & \(\sqrt{2} \basis_2 + \basis_0\)                           & \((2\basis_2 / \sqrt{5} + \basis_0) / 3\) \\
      \(x^3\)             & \(\sqrt{6} \basis_3 + 3\basis_1\)                          & \((2\basis_3 / \sqrt{7} + 3\basis_1 / \sqrt{3}) / 5\) \\
      \(x^4\)             & \(2\sqrt{6} \basis_4 + 6\sqrt{2} \basis_2 + 3\basis_0\)    & \((8\basis_4 / 3 + 20\basis_2 / \sqrt{5} + 7\basis_0) / 35\) \\
      \(x^5\)             & \(2\sqrt{30} \basis_5 + 10\sqrt{6} \basis_3 + 15\basis_1\) & \((8\basis_5 / \sqrt{11} + 28\basis_3 / \sqrt{7} + 27\basis_1 / \sqrt{3}) / 63\) \\
      \bottomrule
    \end{tabular}
\end{table}