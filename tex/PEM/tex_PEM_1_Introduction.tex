%%%%%%%%%%%%%%%%
% INTRODUCTION %
%%%%%%%%%%%%%%%%
Main characteristics and challenges of inverse problems in engineering sciences subsume the following issues.
% FORWARD MODELING
Firstly, the ever-growing complexity of physical modeling increases the computational expense of deterministic forward simulations.
% UNCERTAINTY MODELING
Secondly, uncertainty is omnipresent and calls for an adequate mathematical formalism of representation and management.
% SCARCE DATA
Thirdly, since data are commonly scarce or prohibitively expensive to acquire, the available information has to be carefully handled.
% GENERIC PROBLEM
An abstract inverse problem statement thus reads as follows.
By analyzing a limited amount of data the endeavor is to optimally learn about unknown forward model inputs that are subject to epistemic uncertainty and aleatory variability.
% (HYPER)PARAMETERS
This includes deducing fixed albeit unknown forward model parameters as well as hyperparameters that determine the distribution of variable model inputs.
% NO PREVIOUS SOLUTION
Such a universal formulation describes a class of inverse problems that has hardly been satisfactorily solved yet.
% MOTIVATION
Our goal is therefore to develop a rigorous and extensive framework for formulating and solving such inverse problems in support of data analysis for engineering systems.
% RESEARCH FOCUS
The focus of this research is on experimental situations as they are typically encountered in this field.
We emphasize aspects of uncertainty quantification and information accumulation.
% FRAMEWORK
In order to establish a sound conceptional and computational basis for solving those problems one has to complement ideas and techniques that have been developed in different academic disciplines and scientific communities so far.
% SOURCES
This involves inverse modeling, Bayesian statistics and uncertainty quantification.
In the following we will shortly survey relevant theories and practices.
%%%%%%%%%%%%%%%%%%%%%%%%%
\par % INVERSE PROBLEMS %
%%%%%%%%%%%%%%%%%%%%%%%%%
In the first place we rely on the Bayesian approach to \textit{classical inverse problems} \cite{Bayesian:Stuart2010,Bayesian:Allmaras2013}.
% PROBLEM CHARACTERISTICS
When a physical theory or a computational solver relates physical parameters to measurable quantities, i.e.\ the \textit{forward model},
classical inversion is the process of reasoning or inferring unknown yet physically fixed model parameters from recorded data \cite{Inversion:Tarantola2005,Inversion:Kaipio2005}.
% BAYESIAN INFERENCE
Bayesian inference establishes a convenient probabilistic framework to accomplish this conventional type of parameter estimation and data assimilation.
% ENGINEERING APPLICATIONS
At least since the advent of the personal computer it is nowadays widely used in engineering applications \cite{Bayesian:Hadidi2008,Bayesian:Beck2010}.
% HIERARCHICAL INVERSION
The stochastic paradigm provides a natural mechanism for the regularization of ill-posed problems, however, it requires the specification of a prior and a noise model.
\textit{Hierarchical inversion} is an extension of the classical framework that allows to set parameters of the prior and the noise model in a data-informed manner \cite{Multilevel:Malinverno2004,Inversion:Wang2005:b}.
% CHARACTERISTICS & DEFICIENCIES
While epistemic uncertainty is naturally incorporated, a shortcoming of these types of parameter estimation is that they do not account for aleatory variability.
%%%%%%%%%%%%%%%%%%%%%%%%%%%%
\par % HIERARCHICAL MODELS %
%%%%%%%%%%%%%%%%%%%%%%%%%%%%
In the second place \textit{hierarchical statistical models} serve as the main tool for the analysis of complex systems.
Those are systems that are hierarchically organized at multiple nested layers.
% RANDOM EFFECTS
Prominent instances include \textit{random} and \textit{mixed effects models} \cite{Multilevel:Wu2010}.
% HISTORICALLY SPEAKING
Historically those models were developed in social and biological sciences e.g.\ for purposes of educational research
\cite{Multilevel:Raudenbusch1988,Multilevel:Seltzer1996} and pharmacokinetics/dynamics \cite{Multilevel:Wakefield1996,Multilevel:Banks2004}.
% RECENT REVIEWS
Some recent reviews about the methods that were developed in these fields can be found in \cite{Multilevel:Davidian2003,Multilevel:Banks2012}.
% FREQUENTISM VS. BAYESIANISM
Hierarchical modeling can be viewed from a more frequentist \cite{Multilevel:Davidian1995,Multilevel:Banks2014} or a more Bayesian perspective \cite{Multilevel:Gelman2006:a,Bayesian:Congdon2010}.
% NOWADAYS IMPORTANCE
At the present day it is mature area of research that establishes sort of an overarching theme in modern multidisciplinary statistics.
Dedicated chapters can be found in numerous standard references for Bayesian modeling and inference \cite{Bayesian:Jackman2009,Bayesian:Gelman2014:3rd}.
% CHARACTERISTICS & DEFICIENCIES
A general observation is that hierarchical models may be complex in their probabilistic architecture whereas only little forward modeling takes place.
%%%%%%%%%%%%%%%%%%%%%%%%%%%%%
\par % UNCERTAINTY MODELING %
%%%%%%%%%%%%%%%%%%%%%%%%%%%%%
In the third place we respect the uncertainty taxonomy that is prevalent in risk assessment and decision making.
According to this classification one distinguishes between epistemic and aleatory uncertainty \cite{Uncertainty:Faber2005,Uncertainty:Kiureghian2009}.
% EPISTEMIC UNCERTAINTY
On one side, \textit{epistemic uncertainty} refers to the ignorance or lack of knowledge of the observer and analyst.
By taking further evidence this type of uncertainty is reducible in principle.
% ALEATORY VARIABILITY
On the contrary, \textit{aleatory uncertainty} or \textit{variability} refers to a trait of the system under consideration.
It is a structural randomness of irreducible character.
% UNCERTAINTY REPRESENTATION
Uncertainties can be accounted for in distinct mathematical frameworks and especially the representation of ignorance is the subject matter of ongoing debates \cite{Uncertainty:Helton2004,Uncertainty:Helton2011}.
% BAYESIAN NETWORKS (BPN)
Graphical statistical models such as \textit{Bayesian probability networks} establish a powerful and widespread tool of uncertainty characterization \cite{Bayesian:Koski2009,Bayesian:Kjaerulff2013}.
In risk-based decision making Bayesian belief networks have been adopted for their strength and flexibility in uncertainty modeling \cite{Bayesian:Bayraktarli2011:a,Bayesian:Deublein2013}
and their elegant mechanisms of information aggregation \cite{Bayesian:Kelly2009,Bayesian:Urbina2012}.
%%%%%%%%%%%%%%%%%%%%%%%%%%%%%%%%
\par % PROBABILISTIC INVERSION %
%%%%%%%%%%%%%%%%%%%%%%%%%%%%%%%%
In the fourth place \textit{probabilistic inverse problems} constitute a challenging class of inverse problems that is of theoretical and practical relevance alike.
% INFERENCE OF VARIABILITY
While classical inversion is concerned with estimating uncertain yet physically fixed parameters in a series of experiments, i.e.\ identifying an epistemically uncertain quantity,
probabilistic inversion deals with inferring the distribution of such forward model inputs that vary throughout the experiments, i.e.\ quantifying their aleatory variability.
% LITERATURE REVIEW
Previously established approaches to this interesting type of problems with \textit{latent/hidden variable} structure subsume various approximate solutions.
% LIKELIHOOD MARGINALIZATION
A frequentist technique that is premised on the simulation of an explicitly marginalized likelihood is proposed in \cite{Multilevel:Rocquigny2009}.
% EXPECTATION-MAXIMIZATION (EM)
There are also attempts to compute approximate solutions based on variants of the expectation-maximization algorithm
within a linearized Gaussian frame \cite{Multilevel:Celeux2010} or with the aid of Kriging surrogates \cite{Multilevel:Barbillon2011}.
% OVERVIEW
A methodological review of this school of probabilistic inversion is found in \cite{Uncertainty:Rocquigny2012}.
% CHARACTERISTICS & DEFICIENCIES
These methods are only partly Bayesian and suffer from the deficiency of providing mere point estimates.
%%%%%%%%%%%%%%%%%%%%%%%%%
\par % BRIDGING THE GAP %
%%%%%%%%%%%%%%%%%%%%%%%%%
The potential of hierarchical models as instruments of statistical modeling and uncertainty quantification have barely been acknowledged for the purposes of inversion in a classical sense.
% HIERARCHICAL & PROBABILISTIC INVERSION
Hierarchical and probabilistic inversion are first steps towards preparing the Bayesian framework for the treatment of more realistic experimental scenarios.
These approaches do not fully exhaust the inferential machinery of hierarchical models and the probability logic of Bayesian networks, though.
% OUR CONTRIBUTION
In this contribution we thus aim at bridging that gap by developing a coherent Bayesian framework for managing uncertainties in such undertakings.
% MULTILEVEL FRAMEWORK
By drawing on the statistical theory of hierarchical models, we cast inversion under parameter uncertainty and variability as \textit{Bayesian multilevel calibration}.
% JOINT VS. MARGINAL & NUMERICAL SOLUTIONS 
This embeds a joint and a marginal problem formulation of Bayesian inference under uncertainty, both of which can be numerically solved with plain vanilla or specialized Markov chain Monte Carlo methods.
\par % ENGINEERING APPLICATIONS
This new formulation of \textit{multilevel inversion} is especially well-adapted to the challenges that engineers are frequently faced with.
% UNCERTAINTY MODELING
It naturally allows for sophisticated uncertainty modeling which comprises both epistemic and aleatory uncertainty.
The inclusion of the former is straightforward whereas the introduction of the latter is an extension to classical parameter estimation.
% BLACKBOX POV
It also promotes a pervasive ``blackbox'' point of view on the forward model.
While this is inevitable in many complex applications, it is not readily compliant with traditional hierarchical models.
% SPECIAL CASES & EXTENSIONS
Previously established strategies of enhanced uncertainty quantification, e.g.\ hierarchical and probabilistic inversion, emerge as special cases of the proposed general problem formulation.
% FULLY BAYESIAN PROBABILISTIC INVERSION
This also offers the opportunity to cope with probabilistic inversion within a fully Bayesian setting.
% NEW IDEAS
Beyond these extensions some fundamentally new possibilities are suggested.
% ``PERFECT'' DATA
Based on the probabilistic calculus of multilevel models, we develop a novel formulation of multilevel inversion in the zero-noise and ``perfect'' data limit.
% BORROWING STRENGTH
The statistical effect of ``borrowing strength'' or ``optimal combination of information'' is transferred and applied to inverse problems.
%%%%%%%%%%%%%%%%
\par % OUTLINE %
%%%%%%%%%%%%%%%%
The article is organized as follows.
In \cref{sec:PEM:Multilevel} we will elaborate a general Bayesian framework for the treatment of uncertainty and variability in inverse problems.
This is followed by a discussion about Bayesian inference in the context of multilevel inversion in \cref{sec:PEM:Inference}.
Thereafter \cref{sec:PEM:PerfectData} will provide an extension of the framework that will allow for handling ``perfect'' data.
Probabilistic inversion and borrowing strength will be placed in context in \cref{sec:PEM:ProbInv,sec:PEM:CombInf}, respectively.
Dedicated Bayesian computations based on Markov chain Monte Carlo are reviewed in \cref{sec:PEM:Computations}.
Lastly in \cref{sec:PEM:CaseStudies} we will conduct a selection of numerical case studies, where by considering various experimental situations and uncertainty setups
the very potential and the computational challenges of the devised modeling paradigm will become transparent.