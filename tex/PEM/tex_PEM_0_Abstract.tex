% FRAMEWORK
In this paper a unified probabilistic framework for solving inverse problems in the presence of epistemic and aleatory uncertainty is presented.
% EXPERIMENTAL SITUATIONS
The aim is to establish a flexible theory that facilitates Bayesian data analysis in experimental scenarios as they are commonly met in engineering practice.
% PROBLEM STATEMENT
Problems are addressed where learning about unobservable inputs of a forward model,
e.g.\ reducing the epistemic uncertainty of fixed yet unknown parameters and/or quantifying the aleatory uncertainty of variable inputs,
is based on processing response measurements.
% MAIN SOURCES & % FINAL FRAMEWORK
Approaches to Bayesian inversion, hierarchical modeling and uncertainty quantification are combined into a generic framework
that eventually allows to interpret and accomplish this task as multilevel model calibration.
% EQUIVALENT FORMULATIONS
A joint problem formulation, where quantities that are not of particular interest are marginalized out from a joint posterior distribution,
or an intrinsically marginal formulation, which is based on an integrated likelihood function,
can be chosen according to the inferential objective and computational convenience.
% PROBABILISTIC INVERSION
Fully Bayesian probabilistic inversion, i.e.\ the inference the variability of unobservable model inputs across a number of experiments, is derived as a special case of multilevel inversion.
% BORROWING STRENGTH
Borrowing strength, i.e.\ the optimal estimation of experiment-specific unknown forward model inputs, is introduced as a means for combining information in inverse problems.
% (IM)``PERFECT'' DATA
Two related statistical models for situations involving finite or zero model/measurement error are devised.
% COMPUTATIONAL OBSTACLES
Multilevel-specific obstacles to Bayesian posterior computation via Markov chain Monte Carlo are discussed.
% NUMERICAL DEMONSTRATION
The inferential machinery of Bayesian multilevel model calibration and its underlying flow of information are studied on the basis of a system from the domain of civil engineering.
A population of identically manufactured structural elements serves as an exemplary system for examining different experimental settings from the standpoint of uncertainty quantification and reduction.
% SPECIAL CASES
In a series of tests the material variability throughout the ensemble of specimens, the entirety of specimen-specific material properties
and the measurement error level are inferred under various uncertainties in the problem setup.