%%%%%%%%%%%%%%%%%%%%%%%%%%
% CONCLUSION AND OUTLOOK %
%%%%%%%%%%%%%%%%%%%%%%%%%%
Bayesian multilevel model calibration has been developed as a consistent and comprehensive framework for managing uncertainties in inverse problems.
% COMPLEX INVERSE PROBLEMS
At the core of the such problems a forward model relates physical parameters to observable quantities.
This deterministic model has been surrounded by a probabilistic representation of uncertainty, variability and error.
% COMBINATION OF IDEAS
For this purpose classical Bayesian inversion, hierarchical statistical models and the predominant epistemic/aleatory conception of uncertainty have been utilized.
% INFERENTIAL MECHANISM & RESEARCH FOCUS
The inferential rationale of multilevel inversion, based on the conditioning, marginalization and transformation of probability measures,
has become transparent by laying the research focus on aspects of uncertainty quantification and information accumulation.
% PROBABILISTIC INVERSION & BORROWING STRENGTH
Fully Bayesian probabilistic inversion and borrowing strength have been suggested.
% ``PERFECT'' DATA
Furthermore we have originally elaborated on the ``perfect'' data limit.
% EXPERIMENTAL SCENARIOS
Our developments were driven by the challenges of engineering applications and they ultimately allow for optimal data analysis in intricate situations where evidence is scarce and uncertainty prevails.
\par % NUMERICAL DEMONSTRATIONS
An ensemble of structural elements of the same type, for all of which virtual tests are performed and pseudo data are gathered, served as the basis for investigating a variety of experimental scenarios.
The amenities of Bayesian multilevel inversion were demonstrated by exercising inference in the chosen example applications under realistic uncertainty configurations.
% PROBABILISTIC INVERSION
Probabilistic inversion, i.e.\ the identification of material variability throughout a population of specimens, was accomplished and it was investigated how the amount of data influences the estimation uncertainty .
% GENERALIZATIONS
The constraints of perfectly known residual variances and experimental conditions were loosened.
In this context we calibrated the forward model prediction error and we studied how the objective of probabilistic inversion is impeded by additional uncertainties in the experimental conditions.
% BORROWING STRENGTH
Optimal combination of information, i.e.\ the ideal inference of specimen-specific properties, has been introduced as a byproduct of the joint formulation of multilevel inversion.
Especially in the engineering community this is an aspect that is often overlooked.
% COMPUTIONAL BURDEN
We examined the underlying inferential mechanisms and we identified the computational obstacles, e.g.\ costly evaluations of the marginalized likelihood function or the curse of high-dimensionality.
\par % OUR CONCLUSIONS
In conclusion, innovative techniques must be developed in order to overcome these difficulties for solving ``real-world'' problems.
Future research therefore includes the following items.
% LIKELIHOOD APPROXIMATIONS
For the marginal problem, numerically efficient and acceptably accurate approximations of the integrated likelihood have to be developed.
% ADVANCED MCMC
Advanced MCMC techniques, that are custom-tailored for the specific structure of multilevel posteriors, have to be devised for the joint problem.
In this connection a numerical study involving HMC is in progress.
% METAMODELING
For both the marginal and the joint variant of multilevel inversion, the application of dedicated metamodeling techniques promises drastic speedups.
% OPTIMAL TRANSPORTATION
It will also be interesting to study the applicability and performance of optimal transportation approaches \cite{Mapping:Reich2011,Mapping:ElMoselhy2012} to classical Bayesian inference in the context of multilevel estimation.
% MULTIMODALITY & ILL-POSEDNESS
Another research question concerns the role of multimodality and severe ill-posedness of separate inverse problems in Bayesian multilevel inversion.