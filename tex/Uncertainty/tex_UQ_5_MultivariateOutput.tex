% MULTIVARIATE OUTPUT
Now we consider vector-valued models \(\mathcal{M} \colon \mathcal{D}_{\bm{x}} \rightarrow \mathds{R}^\dimData\) that map inputs
\(\bm{x} \in \mathcal{D}_{\bm{x}}\) to multiple outputs \(\tilde{\bm{y}} = \mathcal{M}(\bm{x}) \in \mathds{R}^\dimData\) with \(\dimData \in \mathds{N}_{>0}\).
Each component of the output vector \(\tilde{\bm{y}} = (\tilde{y}_1,\ldots,\tilde{y}_\dimData)^\top\)
is predicted as \(\tilde{y}_i = \mathcal{M}_{\tilde{y}_i}(\bm{x}) \in \mathds{R}\) for \(i=1,\ldots,\dimData\)
by a function of all inputs denoted as \(\mathcal{M}_{\tilde{y}_i} \colon \mathcal{D}_{\bm{x}} \rightarrow \mathds{R}\).
In explicit notation this can be written as
\begin{equation} \label{eq:UQ:VectorModel}
  \tilde{\bm{y}}
  = \begin{pmatrix}
      \tilde{y}_1 \\
      \vdots \\
      \tilde{y}_\dimData \\
    \end{pmatrix}
  = \begin{pmatrix}
      \mathcal{M}_{\tilde{y}_1}(\bm{x}) \\
      \vdots \\
      \mathcal{M}_{\tilde{y}_\dimData}(\bm{x}) \\
    \end{pmatrix}.
\end{equation}
% VECTOR SPACES
One should distinguish between the predicted output \(\tilde{\bm{y}} \in \mathds{R}^\dimData\) as an element in the vector space \(\mathds{R}^\dimData\)
and the predicting model \(\mathcal{M}_{\tilde{y}_i} \in L_{\pi}^2(\mathcal{D}_{\bm{x}})\) which is assumed to be an element of the function space \(L_{\pi}^2(\mathcal{D}_{\bm{x}})\).
Accordingly, for \(i=1,\ldots,\dimData\) one can represent, project and approximate each \(\mathcal{M}_{\tilde{y}_i}\) separately with the theory and methods previously discussed.
\par % CHANGE OF BASIS
One can also represent the model outputs \(\tilde{\bm{y}} \in \mathds{R}^\dimData\) with respect to another basis of the model output space.
Let \(\{\bm{\phi}_i\}_{i=1}^\dimData\) be an orthonormal basis of the Euclidean vector pace \(\mathds{R}^\dimData\)
which is different from the standard reference system \(\{\bm{e}_i\}_{i=1}^\dimData\).
As an alternative to the naturally suggested representation \(\tilde{\bm{y}} = \sum_{i=1}^\dimData \tilde{y}_i \bm{e}_i\), the model responses can be expanded as
\begin{equation} \label{eq:UQ:ChangeOfBasis}
  \tilde{\bm{y}} = \sum\limits_{i=1}^\dimData \tilde{z}_i \bm{\phi}_i.
\end{equation}
Here, \(\tilde{\bm{z}} = (\tilde{z}_1,\ldots,\tilde{z}_\dimData)^\top\) is the coordinate vector of the output relative to the alternative basis,
where each coordinate is given as \(\tilde{z}_i = \tilde{\bm{y}}^\top \bm{\phi}_i = \tilde{\bm{y}} \cdot \bm{\phi}_i\) for \(i=1,\ldots,\dimData\).
As a function of the uncertain model parameters, each \(\tilde{z}_i = \mathcal{M}_{\tilde{z}_i}(\bm{x})\)
is predicted by the corresponding scalar-valued function \(\mathcal{M}_{\tilde{z}_i} \in L_{\pi}^2(\mathcal{D}_{\bm{x}})\).
\par % OPTIMALITY
Thus far, the canonical representation in \cref{eq:UQ:VectorModel} has been merely reformulated in \cref{eq:UQ:ChangeOfBasis}.
Even though the two formulations before and after the change a basis are technically equivalent,
they may differ in their eligibility for signal compaction \cite{Statistics:Ahmed1975,Statistics:Wang2012}.
If the output dimension is high, it is inconvenient at the very least to have to treat multiple outputs individually.
Moreover, different model outputs are often redundant to some degree anyhow.
This motivates to consider bases where the essential features of the model are captured by just a few dominant terms.
Such considerations are especially important against the backdrop of the following section.