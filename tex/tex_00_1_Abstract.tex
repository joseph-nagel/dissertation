% BAYESIAN INFERENCE
Bayesian inference enables the fusion of heterogeneous information and the reduction of epistemic uncertainty for solving inverse problems.
Physical models, expert knowledge and experimental data are statistically interpreted in order to learn about the unknown model parameters.
The prior and the posterior probability distribution represent the uncertainty of the unknowns before and after the analysis.
Bayes' theorem governs the update from the prior to the conditional posterior which reflects the achieved gain of information.
\par % POSTERIOR COMPUTATION
Characterizing the posterior distribution poses the main challenge in Bayesian data analysis.
Since only very simple problems admit analytical solutions, most often one has to compute the posterior numerically.
This formidable task is accomplished by means of Markov chain Monte Carlo techniques.
The large number of forward model runs that is thereby required prohibits computational inference in many fields of application.
In civil, mechanical and aerospace engineering this holds especially true.
\par % CORE CONTRIBUTIONS
The goal of this doctoral dissertation is to develop new approaches to the Bayesian probabilistic analysis in complex and realistic engineering applications.
A unified framework for inverse problems under epistemic uncertainty and aleatory variability is elaborated to that end.
Hamiltonian Monte Carlo is used as an efficient sampling algorithm that overcomes the associated computational difficulties.
Moreover, completely novel methods for posterior computation are presented and investigated.